%! suppress = MissingLabel
\documentclass[12pt,letterpaper,oneside,reqno]{amsart}
\usepackage{amsfonts}
\usepackage{amsmath}
\usepackage{amssymb}
\usepackage{amsthm}
\usepackage{float}
\usepackage{mathrsfs}
\usepackage{colonequals}
\usepackage[font=small,labelfont=bf]{caption}
\usepackage[left=1in,right=1in,bottom=1in,top=1in]{geometry}
\usepackage[pdfpagelabels,hyperindex,colorlinks=true,linkcolor=blue,urlcolor=magenta,citecolor=green]{hyperref}
\usepackage{graphicx}
\linespread{1.7}
\emergencystretch=1em
\usepackage{array}
\usepackage{etoolbox}
\apptocmd{\sloppy}{\hbadness 10000\relax}{}{}
\raggedbottom

\newcommand \anglePower [2]{\langle #1 \rangle \sp{#2}}
\newcommand \bernoulli [2][B] {{#1}\sb{#2}}
\newcommand \curvePower [2]{\{#1\}\sp{#2}}
\newcommand \coeffA [3][A] {{\mathbf{#1}} \sb{#2,#3}}
\newcommand \polynomialP [4][P]{{#1}(#2,#3,#4)}
\newcommand \polynomialQ [4][Q]{{#1}(#2,#3,#4)}

% ~~~ Rascal numbers ~~~
%\newcommand \rascalNumber [3] {\binom{#1}{#2}_{#3}}
%\newcommand \north[0] {\mathbf{North}}
%\newcommand \south[0] {\mathbf{South}}
%\newcommand \west[0] {\mathbf{West}}
%\newcommand \east[0] {\mathbf{East}}

% ~~~~ 1-q pascal notation~~~~
%\newcommand{\genstirlingI}[3]{%
%    \genfrac{[}{]}{0pt}{#1}{#2}{#3}%
%}
%\newcommand{\genstirlingII}[3]{%
%    \genfrac{\{}{\}}{0pt}{#1}{#2}{#3}%
%}
%\newcommand{\oneQBinomial}[3]{\genstirlingI{}{#1}{#2}^{#3}}

% free foot note
%\let\svthefootnote\thefootnote
%\newcommand\freefootnote[1]{%
%    \let\thefootnote\relax%
%    \footnotetext{#1}%
%    \let\thefootnote\svthefootnote%
%}


\newtheorem{theorem}{Theorem}[section]
\newtheorem{corollary}[theorem]{Corollary}
\newtheorem{lemma}[theorem]{Lemma}
\newtheorem{example}[theorem]{Example}
\newtheorem{conjecture}[theorem]{Conjecture}
\newtheorem{definition}[theorem]{Definition}

%\numberwithin{equation}{section}

\title[Plots of Closed Forms]
{Plots of Closed Forms}
\author[Petro Kolosov]{Petro Kolosov}
%\address{Software Developer, DevOps Engineer}
%\email{kolosovp94@gmail.com}
%\urladdr{https://kolosovpetro.github.io}
%\keywords{
%    Binomial theorem,
%    Binomial coefficients,
%    Faulhaber's formula,
%    Polynomials,
%    Pascal's triangle
%    Finite differences,
%    Interpolation,
%    Polynomial identities
%}
%\subjclass[2010]{26E70, 05A30}
%\date{\today}
\hypersetup{
    pdftitle={Plots of Closed Forms},
    pdfsubject={
        Polynomials,
        Finite differences,
        Interpolation,
        Approximation,
        Polynomial identities,
        Power sums,
        Binomial theorem,
        Power function,
        Binomial coefficients,
        Bernoulli numbers,
        Pascal's triangle,
        Faulhaber's formula,
        Derivatives,
        Differential calculus,
        Partial differential equations,
        OEIS,
        Bernoulli polynomials,
        Combinatorics,
        Discrete convolution,
        Dynamic systems,
        Time scales
    },
    pdfauthor={Petro Kolosov},
    pdfkeywords={
        Polynomials,
        Finite differences,
        Interpolation,
        Approximation,
        Polynomial identities,
        Power sums,
        Binomial theorem,
        Power function,
        Binomial coefficients,
        Bernoulli numbers,
        Pascal's triangle,
        Faulhaber's formula,
        Derivatives,
        Differential calculus,
        Partial differential equations,
        OEIS,
        Bernoulli polynomials,
        Combinatorics,
        Discrete convolution,
        Dynamic systems,
        Time scales
    }
}
\begin{document}
%    \begin{abstract}
%        \input{sections/01_abstract}
%    \end{abstract}

    \maketitle

    \tableofcontents

%    \freefootnote{Sources: \url{https://github.com/kolosovpetro/github-latex-template}}

    \section{Introduction}\label{sec:introduction}
    \begin{align*}
        \polynomialP{m}{X}{N} &= \sum_{r=0}^{m} \sum_{k=1}^{N} \coeffA{m}{r} k^r (X-k)^r \\
        \polynomialP{m}{X}{N} &= \sum_{r=0}^{m} (-1)^{m-r} U(m, N, r) \cdot X^{r} \\
        \polynomialQ{m}{X}{N} &= \sum_{r=0}^{m} \sum_{k=0}^{N-1} \coeffA{m}{r} k^r (X-k)^r \\
        \polynomialQ{m}{X}{N} &= \sum_{r=0}^{m} (-1)^{m-r} V(m, N, r) \cdot X^{r}
    \end{align*}

    \begin{align*}
        \polynomialP{m}{N}{N} &= N^{2m+1} \\
        \polynomialQ{m}{N}{N} &= N^{2m+1} \\
        \polynomialP{m}{N+1}{N} &= (N+1)^{2m+1} - 1 \quad \quad (verified) \\
        \polynomialQ{m}{N-1}{N} &= (N-1)^{2m+1} + 1 \quad \quad (verified)
    \end{align*}

    % P1
    \subsection{Polynomials P(1,X,N)}
    \input{sections/011_polynomials_p1}

    \subsection{Polynomial P(1,X,N) Table of values for N = 6}
    \begin{table}[h!]
    \centering
    \caption{Comparison of $X^3$, $\polynomialP{1}{X}{6} =126X-540$, Absolute, Relative, and Percentage Error}
    \begin{tabular}{|c|c|c|c|c|c|}
        \hline
        \textbf{X} & \textbf{$X^3$} & \textbf{$126X-540$} & \textbf{ABS} & \textbf{Relative} & \textbf{\% Error} \\ \hline
        5.3        & 148.877        & 127.8               & 21.077       & 0.141573          & 14.1573           \\ \hline
        5.4        & 157.464        & 140.4               & 17.064       & 0.108368          & 10.8368           \\ \hline
        5.5        & 166.375        & 153.0               & 13.375       & 0.0803907         & 8.03907           \\ \hline
        5.6        & 175.616        & 165.6               & 10.016       & 0.0570335         & 5.70335           \\ \hline
        5.7        & 185.193        & 178.2               & 6.993        & 0.0377606         & 3.77606           \\ \hline
        5.8        & 195.112        & 190.8               & 4.312        & 0.0221001         & 2.21001           \\ \hline
        5.9        & 205.379        & 203.4               & 1.979        & 0.00963584        & 0.963584          \\ \hline
        6.0        & 216.0          & 216.0               & 0.0          & 0.0               & 0.0               \\ \hline
        6.1        & 226.981        & 228.6               & 1.619        & 0.00713276        & 0.713276          \\ \hline
        6.2        & 238.328        & 241.2               & 2.872        & 0.0120506         & 1.20506           \\ \hline
        6.3        & 250.047        & 253.8               & 3.753        & 0.0150092         & 1.50092           \\ \hline
        6.4        & 262.144        & 266.4               & 4.256        & 0.0162354         & 1.62354           \\ \hline
        6.5        & 274.625        & 279.0               & 4.375        & 0.0159308         & 1.59308           \\ \hline
        6.6        & 287.496        & 291.6               & 4.104        & 0.014275          & 1.4275            \\ \hline
        6.7        & 300.763        & 304.2               & 3.437        & 0.0114276         & 1.14276           \\ \hline
        6.8        & 314.432        & 316.8               & 2.368        & 0.00753104        & 0.753104          \\ \hline
        6.9        & 328.509        & 329.4               & 0.891        & 0.00271225        & 0.271225          \\ \hline
        7.0        & 343.0          & 342.0               & 1.0          & 0.00291545        & 0.291545          \\ \hline
        7.1        & 357.911        & 354.6               & 3.311        & 0.0092509         & 0.92509           \\ \hline
        7.2        & 373.248        & 367.2               & 6.048        & 0.0162037         & 1.62037           \\ \hline
        7.3        & 389.017        & 379.8               & 9.217        & 0.0236931         & 2.36931           \\ \hline
        7.4        & 405.224        & 392.4               & 12.824       & 0.0316467         & 3.16467           \\ \hline
        7.5        & 421.875        & 405.0               & 16.875       & 0.04              & 4.0               \\ \hline
        7.6        & 438.976        & 417.6               & 21.376       & 0.0486951         & 4.86951           \\ \hline
        7.7        & 456.533        & 430.2               & 26.333       & 0.0576804         & 5.76804           \\ \hline
        7.8        & 474.552        & 442.8               & 31.752       & 0.0669094         & 6.69094           \\ \hline
        7.9        & 493.039        & 455.4               & 37.639       & 0.0763408         & 7.63408           \\ \hline
        8.0        & 512.0          & 468.0               & 44.0         & 0.0859375         & 8.59375           \\ \hline
        8.1        & 531.441        & 480.6               & 50.841       & 0.0956663         & 9.56663           \\ \hline
        8.2        & 551.368        & 493.2               & 58.168       & 0.105498          & 10.5498           \\ \hline
    \end{tabular}\label{tab:table}
\end{table}


    \subsection{Polynomial P(1,X,6) plot with cubes}
    \begin{figure}[H]
    \centering
    \includegraphics[width=1\textwidth]{sections/images/01_plots_polynomial_p1_n6_with_cubes}
    ~\caption{Polynomial plot $P(1, X, 6)$ with cubes $X^3$.
    Points of intersection $X=6$, $X=6.94987$.
    Interval of convergence: $5.9 \leq X \leq 7.2$.
    }\label{fig:figure7}
\end{figure}


    % Q1
    \subsection{Polynomials Q(1,X,N)}
    \input{sections/021_polynomials_q1}

    \subsection{Polynomial Q(1,X,N) Table of values for N = 6}
    \begin{table}[h!]
    \centering
    \caption{Comparison of $X^3$, $\polynomialQ{1}{X}{6} = 90X-324$, Absolute, Relative, and Percentage Error}
    \begin{tabular}{|c|c|c|c|c|c|}
        \hline
        \textbf{X} & \textbf{$X^3$} & \textbf{$90X-324$} & \textbf{ABS} & \textbf{Relative} & \textbf{\% Error} \\ \hline
        4.5        & 91.125         & 81.0               & 10.125       & 0.111111          & 11.1111           \\ \hline
        4.6        & 97.336         & 90.0               & 7.336        & 0.0753678         & 7.53678           \\ \hline
        4.7        & 103.823        & 99.0               & 4.823        & 0.0464541         & 4.64541           \\ \hline
        4.8        & 110.592        & 108.0              & 2.592        & 0.0234375         & 2.34375           \\ \hline
        4.9        & 117.649        & 117.0              & 0.649        & 0.00551641        & 0.551641          \\ \hline
        5.0        & 125.0          & 126.0              & 1.0          & 0.008             & 0.8               \\ \hline
        5.1        & 132.651        & 135.0              & 2.349        & 0.0177081         & 1.77081           \\ \hline
        5.2        & 140.608        & 144.0              & 3.392        & 0.0241238         & 2.41238           \\ \hline
        5.3        & 148.877        & 153.0              & 4.123        & 0.027694          & 2.7694            \\ \hline
        5.4        & 157.464        & 162.0              & 4.536        & 0.0288066         & 2.88066           \\ \hline
        5.5        & 166.375        & 171.0              & 4.625        & 0.0277986         & 2.77986           \\ \hline
        5.6        & 175.616        & 180.0              & 4.384        & 0.0249636         & 2.49636           \\ \hline
        5.7        & 185.193        & 189.0              & 3.807        & 0.0205569         & 2.05569           \\ \hline
        5.8        & 195.112        & 198.0              & 2.888        & 0.0148018         & 1.48018           \\ \hline
        5.9        & 205.379        & 207.0              & 1.621        & 0.00789273        & 0.789273          \\ \hline
        6.0        & 216.0          & 216.0              & 0.0          & 0.0               & 0.0               \\ \hline
        6.1        & 226.981        & 225.0              & 1.981        & 0.0087276         & 0.87276           \\ \hline
        6.2        & 238.328        & 234.0              & 4.328        & 0.0181598         & 1.81598           \\ \hline
        6.3        & 250.047        & 243.0              & 7.047        & 0.0281827         & 2.81827           \\ \hline
        6.4        & 262.144        & 252.0              & 10.144       & 0.0386963         & 3.86963           \\ \hline
        6.5        & 274.625        & 261.0              & 13.625       & 0.0496131         & 4.96131           \\ \hline
        6.6        & 287.496        & 270.0              & 17.496       & 0.0608565         & 6.08565           \\ \hline
        6.7        & 300.763        & 279.0              & 21.763       & 0.0723593         & 7.23593           \\ \hline
        6.8        & 314.432        & 288.0              & 26.432       & 0.0840627         & 8.40627           \\ \hline
        6.9        & 328.509        & 297.0              & 31.509       & 0.0959152         & 9.59152           \\ \hline
        7.0        & 343.0          & 306.0              & 37.0         & 0.107872          & 10.7872           \\ \hline
    \end{tabular}\label{tab:table4}
\end{table}


    \subsection{Polynomial Q(1,X,6) plot with cubes}
    \begin{figure}[H]
    \centering
    \includegraphics[width=1\textwidth]{sections/images/02_plots_polynomial_q1_n6_with_cubes}
    ~\caption{Polynomial plot Q(1, X, 6) with cubes.
    Points of intersection: $X=6$, $X=4.93725$.}\label{fig:figure8}
\end{figure}


    % P2
    \subsection{Polynomials P(2,X,N)}
    \begin{align*}
    \polynomialP{2}{X}{0} &= 0 \\
    \polynomialP{2}{X}{1} &= 30X^2 - 60X + 31 \\
    \polynomialP{2}{X}{2} &= 150X^2 - 540X + 512 \\
    \polynomialP{2}{X}{3} &= 420X^2 - 2160X + 2943 \\
    \polynomialP{2}{X}{4} &= 900X^2 - 6000X + 10624 \\
    \polynomialP{2}{X}{5} &= 1650X^2 - 13500X + 29375 \\
    \polynomialP{2}{X}{6} &= 2730X^2 - 26460X + 68256 \\
    \polynomialP{2}{X}{7} &= 4200X^2 - 47040X + 140287 \\
    \polynomialP{2}{X}{8} &= 6120X^2 - 77760X + 263168 \\
    \polynomialP{2}{X}{9} &= 8550X^2 - 121500X + 459999 \\
    \polynomialP{2}{X}{10} &= 11550X^2 - 181500X + 760000 \\
    \polynomialP{2}{X}{11} &= 15180X^2 - 261360X + 1199231 \\
    \polynomialP{2}{X}{12} &= 19500X^2 - 365040X + 1821312 \\
    \polynomialP{2}{X}{13} &= 24570X^2 - 496860X + 2678143 \\
    \polynomialP{2}{X}{14} &= 30450X^2 - 661500X + 3830624 \\
    \polynomialP{2}{X}{15} &= 37200X^2 - 864000X + 5349375 \\
    \polynomialP{2}{X}{16} &= 44880X^2 - 1109760X + 7315456 \\
    \polynomialP{2}{X}{17} &= 53550X^2 - 1404540X + 9821087 \\
    \polynomialP{2}{X}{18} &= 63270X^2 - 1754460X + 12970368 \\
    \polynomialP{2}{X}{19} &= 74100X^2 - 2166000X + 16879999 \\
    \polynomialP{2}{X}{20} &= 86100X^2 - 2646000X + 21680000
\end{align*}
\begin{figure}[H]
    \centering
    \includegraphics[width=1\textwidth]{sections/images/03_plots_fifth_with_p2}
    ~\caption{Polynomials P(2, n, k)}\label{fig:figure3}
\end{figure}


    \subsection{Polynomial P(2,X,N) Table of values for N = 4}
    \begin{table}[h!]
    \centering
    \caption{Comparison of $X^5$, $\polynomialP{2}{X}{4} = 900X^2 - 6000X + 10624$, Absolute, Relative, and Percentage Error}
    \begin{tabular}{|c|c|c|c|c|c|}
        \hline
        \textbf{X} & \textbf{$X^5$} & \textbf{$900X^2 - 6000X + 10624$} & \textbf{ABS} & \textbf{Relative} & \textbf{\% Error} \\ \hline
        3.6        & 604.662        & 688.0                             & 83.3382      & 0.137826          & 13.7826           \\ \hline
        3.7        & 693.44         & 745.0                             & 51.5604      & 0.0743546         & 7.43546           \\ \hline
        3.8        & 792.352        & 820.0                             & 27.6483      & 0.034894          & 3.4894            \\ \hline
        3.9        & 902.242        & 913.0                             & 10.758       & 0.0119236         & 1.19236           \\ \hline
        4.0        & 1024.0         & 1024.0                            & 0.0          & 0.0               & 0.0               \\ \hline
        4.1        & 1158.56        & 1153.0                            & 5.56201      & 0.00480079        & 0.480079          \\ \hline
        4.2        & 1306.91        & 1300.0                            & 6.91232      & 0.00528905        & 0.528905          \\ \hline
        4.3        & 1470.08        & 1465.0                            & 5.08443      & 0.0034586         & 0.34586           \\ \hline
        4.4        & 1649.16        & 1648.0                            & 1.16224      & 0.000704746       & 0.0704746         \\ \hline
        4.5        & 1845.28        & 1849.0                            & 3.71875      & 0.00201528        & 0.201528          \\ \hline
        4.6        & 2059.63        & 2068.0                            & 8.37024      & 0.00406395        & 0.406395          \\ \hline
        4.7        & 2293.45        & 2305.0                            & 11.5499      & 0.00503605        & 0.503605          \\ \hline
        4.8        & 2548.04        & 2560.0                            & 11.9603      & 0.00469393        & 0.469393          \\ \hline
        4.9        & 2824.75        & 2833.0                            & 8.24751      & 0.00291973        & 0.291973          \\ \hline
        5.0        & 3125.0         & 3124.0                            & 1.0          & 0.00032           & 0.032             \\ \hline
        5.1        & 3450.25        & 3433.0                            & 17.2525      & 0.00500036        & 0.500036          \\ \hline
        5.2        & 3802.04        & 3760.0                            & 42.0403      & 0.0110573         & 1.10573           \\ \hline
        5.3        & 4181.95        & 4105.0                            & 76.9549      & 0.0184017         & 1.84017           \\ \hline
        5.4        & 4591.65        & 4468.0                            & 123.65       & 0.0269294         & 2.69294           \\ \hline
        5.5        & 5032.84        & 4849.0                            & 183.844      & 0.0365288         & 3.65288           \\ \hline
        5.6        & 5507.32        & 5248.0                            & 259.318      & 0.047086          & 4.7086            \\ \hline
        5.7        & 6016.92        & 5665.0                            & 351.921      & 0.0584885         & 5.84885           \\ \hline
        5.8        & 6563.57        & 6100.0                            & 463.568      & 0.0706274         & 7.06274           \\ \hline
        5.9        & 7149.24        & 6553.0                            & 596.243      & 0.0833995         & 8.33995           \\ \hline
        6.0        & 7776.0         & 7024.0                            & 752.0        & 0.0967078         & 9.67078           \\ \hline
        6.1        & 8445.96        & 7513.0                            & 932.963      & 0.110463          & 11.0463           \\ \hline
    \end{tabular}\label{tab:table2}
\end{table}


    \subsection{Polynomial P(2,X,4) plot with fifth}
    \begin{figure}[H]
    \centering
    \includegraphics[width=1\textwidth]{sections/images/03_plots_polynomial_p2_n4_with_fifth}
    ~\caption{Polynomial plot P(2, X, 4) with fifth.
    Points of intersection $X=4$, $X=4.42472$, $X=4.99181$.}\label{fig:figure9}
\end{figure}


    % Q2
    \subsection{Polynomials Q(2,X,N)}
    \input{sections/041_polynomials_q2}

    \subsection{Polynomial Q(2,X,N) Table of values for N = 4}
    \begin{table}[h!]
    \centering
    \caption{Comparison of $X^5$, $\polynomialQ{2}{X}{4} = 420X^2 - 2160X + 2944$, Absolute, Relative, and Percentage Error}
    \begin{tabular}{|c|c|c|c|c|c|}
        \hline
        \textbf{X} & \textbf{$X^5$} & \textbf{$420X^2-2160X+2944$} & \textbf{ABS} & \textbf{Relative} & \textbf{\% Error} \\ \hline
        2.7        & 143.489        & 173.8                        & 30.3109      & 0.211242          & 21.1242           \\ \hline
        2.8        & 172.104        & 188.8                        & 16.6963      & 0.0970131         & 9.70131           \\ \hline
        2.9        & 205.111        & 212.2                        & 7.08851      & 0.0345593         & 3.45593           \\ \hline
        3.0        & 243.0          & 244.0                        & 1.0          & 0.00411523        & 0.411523          \\ \hline
        3.1        & 286.292        & 284.2                        & 2.09151      & 0.00730553        & 0.730553          \\ \hline
        3.2        & 335.544        & 332.8                        & 2.74432      & 0.00817871        & 0.817871          \\ \hline
        3.3        & 391.354        & 389.8                        & 1.55393      & 0.00397065        & 0.397065          \\ \hline
        3.4        & 454.354        & 455.2                        & 0.84576      & 0.00186146        & 0.186146          \\ \hline
        3.5        & 525.219        & 529.0                        & 3.78125      & 0.00719938        & 0.719938          \\ \hline
        3.6        & 604.662        & 611.2                        & 6.53824      & 0.0108131         & 1.08131           \\ \hline
        3.7        & 693.44         & 701.8                        & 8.36043      & 0.0120565         & 1.20565           \\ \hline
        3.8        & 792.352        & 800.8                        & 8.44832      & 0.0106623         & 1.06623           \\ \hline
        3.9        & 902.242        & 908.2                        & 5.95801      & 0.00660356        & 0.660356          \\ \hline
        4.0        & 1024.0         & 1024.0                       & 0.0          & 0.0               & 0.0               \\ \hline
        4.1        & 1158.56        & 1148.2                       & 10.362       & 0.00894385        & 0.894385          \\ \hline
        4.2        & 1306.91        & 1280.8                       & 26.1123      & 0.0199802         & 1.99802           \\ \hline
        4.3        & 1470.08        & 1421.8                       & 48.2844      & 0.0328447         & 3.28447           \\ \hline
        4.4        & 1649.16        & 1571.2                       & 77.9622      & 0.0472738         & 4.72738           \\ \hline
        4.5        & 1845.28        & 1729.0                       & 116.281      & 0.0630155         & 6.30155           \\ \hline
        4.6        & 2059.63        & 1895.2                       & 164.43       & 0.0798346         & 7.98346           \\ \hline
        4.7        & 2293.45        & 2069.8                       & 223.65       & 0.0975169         & 9.75169           \\ \hline
        4.8        & 2548.04        & 2252.8                       & 295.24       & 0.115869          & 11.5869           \\ \hline
    \end{tabular}\label{tab:table5}
\end{table}


    \subsection{Polynomial Q(2,X,4) plot with fifth}
    \begin{figure}[H]
    \centering
    \includegraphics[width=1\textwidth]{sections/images/04_plots_polynomial_q2_n4_with_fifth}
    ~\caption{Polynomial plot Q(2, X, 4) with fifth}\label{fig:figure10}
\end{figure}


    % P3
    \subsection{Polynomials P(3,X,N)}
    \input{sections/051_polynomials_p3}

    \subsection{Polynomial P(3,X,N) Table of values for N = 3}
    \input{sections/052_polynomials_p3_table_n3}

    \subsection{Polynomial P(3,X,3) plot with seventh}
    \begin{figure}[H]
    \centering
    \includegraphics[width=1\textwidth]{sections/images/05_plots_polynomial_p3_n3_with_seventh}
    ~\caption{Polynomial plot P(3, X, 3) with seventh power $X^7$.
    Points of intersection $X=2.87643$, $X=3$, $X=3.89662$, $X=3.99457$.
    Interval of convergence: $a \leq X \leq b$.
    }\label{fig:figure11}
\end{figure}


    % Q3
    \subsection{Polynomials Q(3,X,N)}
    \input{sections/061_polynomials_q3}

    \subsection{Polynomial Q(3,X,N) Table of values for N = 3}
    \input{sections/062_polynomials_q3_table_n3}

    \subsection{Polynomial Q(3,X,3) plot with seventh}
    \begin{figure}[H]
    \centering
    \includegraphics[width=1\textwidth]{sections/images/06_plots_polynomial_q3_n3_with_seventh}
    ~\caption{Polynomial plot $Q(3, X, 3)$ with seventh power $X^7$.
    Points of intersection $X=1.80948$, $X=2.01364$, $X=2.84612$, $X=3$.
    Interval of convergence: $a \leq X \leq b$.
    }\label{fig:figure12}
\end{figure}



%    \bibliographystyle{unsrt}
%    \bibliography{PlotsOfClosedForms}

\end{document}
