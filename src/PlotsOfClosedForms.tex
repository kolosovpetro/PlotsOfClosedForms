%! suppress = MissingLabel
\documentclass[12pt,letterpaper,oneside,reqno]{amsart}
\usepackage{amsfonts}
\usepackage{amsmath}
\usepackage{amssymb}
\usepackage{amsthm}
\usepackage{float}
\usepackage{mathrsfs}
\usepackage{colonequals}
\usepackage[font=small,labelfont=bf]{caption}
\usepackage[left=1in,right=1in,bottom=1in,top=1in]{geometry}
\usepackage[pdfpagelabels,hyperindex,colorlinks=true,linkcolor=blue,urlcolor=magenta,citecolor=green]{hyperref}
\usepackage{graphicx}
\linespread{1.7}
\emergencystretch=1em
\usepackage{array}
\usepackage{etoolbox}
\apptocmd{\sloppy}{\hbadness 10000\relax}{}{}
\raggedbottom

\newcommand \anglePower [2]{\langle #1 \rangle \sp{#2}}
\newcommand \bernoulli [2][B] {{#1}\sb{#2}}
\newcommand \curvePower [2]{\{#1\}\sp{#2}}
\newcommand \coeffA [3][A] {{\mathbf{#1}} \sb{#2,#3}}
\newcommand \polynomialP [4][P]{{#1}(#2,#3,#4)}
\newcommand \polynomialQ [4][Q]{{#1}(#2,#3,#4)}

% ~~~ Rascal numbers ~~~
%\newcommand \rascalNumber [3] {\binom{#1}{#2}_{#3}}
%\newcommand \north[0] {\mathbf{North}}
%\newcommand \south[0] {\mathbf{South}}
%\newcommand \west[0] {\mathbf{West}}
%\newcommand \east[0] {\mathbf{East}}

% ~~~~ 1-q pascal notation~~~~
%\newcommand{\genstirlingI}[3]{%
%    \genfrac{[}{]}{0pt}{#1}{#2}{#3}%
%}
%\newcommand{\genstirlingII}[3]{%
%    \genfrac{\{}{\}}{0pt}{#1}{#2}{#3}%
%}
%\newcommand{\oneQBinomial}[3]{\genstirlingI{}{#1}{#2}^{#3}}

% free foot note
%\let\svthefootnote\thefootnote
%\newcommand\freefootnote[1]{%
%    \let\thefootnote\relax%
%    \footnotetext{#1}%
%    \let\thefootnote\svthefootnote%
%}


\newtheorem{theorem}{Theorem}[section]
\newtheorem{corollary}[theorem]{Corollary}
\newtheorem{lemma}[theorem]{Lemma}
\newtheorem{example}[theorem]{Example}
\newtheorem{conjecture}[theorem]{Conjecture}
\newtheorem{definition}[theorem]{Definition}

%\numberwithin{equation}{section}

\title[Plots of Closed Forms]
{Plots of Closed Forms}
\author[Petro Kolosov]{Petro Kolosov}
%\address{Software Developer, DevOps Engineer}
%\email{kolosovp94@gmail.com}
%\urladdr{https://kolosovpetro.github.io}
%\keywords{
%    Binomial theorem,
%    Binomial coefficients,
%    Faulhaber's formula,
%    Polynomials,
%    Pascal's triangle
%    Finite differences,
%    Interpolation,
%    Polynomial identities
%}
%\subjclass[2010]{26E70, 05A30}
%\date{\today}
\hypersetup{
    pdftitle={LaTeX Template for Github},
    pdfsubject={
        Polynomials,
        Finite differences,
        Interpolation,
        Approximation,
        Polynomial identities,
        Power sums,
        Binomial theorem,
        Power function,
        Binomial coefficients,
        Bernoulli numbers,
        Pascal's triangle,
        Faulhaber's formula,
        Derivatives,
        Differential calculus,
        Partial differential equations,
        OEIS,
        Bernoulli polynomials,
        Combinatorics,
        Discrete convolution,
        Dynamic systems,
        Time scales
    },
    pdfauthor={Petro Kolosov},
    pdfkeywords={
        Polynomials,
        Finite differences,
        Interpolation,
        Approximation,
        Polynomial identities,
        Power sums,
        Binomial theorem,
        Power function,
        Binomial coefficients,
        Bernoulli numbers,
        Pascal's triangle,
        Faulhaber's formula,
        Derivatives,
        Differential calculus,
        Partial differential equations,
        OEIS,
        Bernoulli polynomials,
        Combinatorics,
        Discrete convolution,
        Dynamic systems,
        Time scales
    }
}
\begin{document}
%    \begin{abstract}
%        \input{sections/01_abstract}
%    \end{abstract}

    \maketitle

    \tableofcontents

%    \freefootnote{Sources: \url{https://github.com/kolosovpetro/github-latex-template}}

    \section{Introduction}\label{sec:introduction}
    \begin{align*}
        \polynomialP{m}{X}{N} &= \sum_{r=0}^{m} \sum_{k=1}^{N} \coeffA{m}{r} k^r (X-k)^r \\
        \polynomialQ{m}{X}{N} &= \sum_{r=0}^{m} \sum_{k=0}^{N-1} \coeffA{m}{r} k^r (X-k)^r
    \end{align*}

    \begin{align*}
        \polynomialP{m}{N}{N} &= N^{2m+1} \\
        \polynomialQ{m}{N}{N} &= N^{2m+1} \\
        \polynomialP{m}{N+1}{N} &= (N+1)^{2m+1} - 1 \quad \quad (verified) \\
        \polynomialQ{m}{N-1}{N} &= (N-1)^{2m+1} + 1 \quad \quad (verified)
    \end{align*}

    \subsection{Polynomials P(1,n,k)}
    \input{sections/01_polynomials_p_1_n_k}

    \subsection{Polynomials P(1,n,k) example n = 6}
    % Table with values of X, X^7, and 126X-540
\begin{table}[h!]
    \centering
    \caption{Values of $X^3$ and $126X - 540$ for selected $X$}
    \begin{tabular}{|c|c|c|}
        \hline
        \textbf{X} & \textbf{$X^7$} & \textbf{$126X - 540$} \\ \hline
        5.0        & 125.000        & 90.000                \\ \hline
        5.1        & 132.651        & 102.600               \\ \hline
        5.2        & 140.608        & 115.200               \\ \hline
        5.3        & 148.877        & 127.800               \\ \hline
        5.4        & 157.464        & 140.400               \\ \hline
        5.5        & 166.375        & 153.000               \\ \hline
        5.6        & 175.616        & 165.600               \\ \hline
        5.7        & 185.193        & 178.200               \\ \hline
        5.8        & 195.112        & 190.800               \\ \hline
        5.9        & 205.379        & 203.400               \\ \hline
        6.0        & 216.000        & 216.000               \\ \hline
        6.1        & 226.981        & 228.600               \\ \hline
        6.2        & 238.328        & 241.200               \\ \hline
        6.3        & 250.047        & 253.800               \\ \hline
        6.4        & 262.144        & 266.400               \\ \hline
        6.5        & 274.625        & 279.000               \\ \hline
        6.6        & 287.496        & 291.600               \\ \hline
        6.7        & 300.763        & 304.200               \\ \hline
        6.8        & 314.432        & 316.800               \\ \hline
        6.9        & 328.509        & 329.400               \\ \hline
        7.0        & 343.000        & 342.000               \\ \hline
        7.1        & 357.911        & 354.600               \\ \hline
        7.2        & 373.248        & 367.200               \\ \hline
        7.3        & 389.017        & 379.800               \\ \hline
        7.4        & 405.224        & 392.400               \\ \hline
        7.5        & 421.875        & 405.000               \\ \hline
    \end{tabular}\label{tab:table}
\end{table}


    \subsection{Polynomials Q(1,n,k)}
    \begin{align*}
    \polynomialQ{1}{N}{0} &= 0 \\
    \polynomialQ{1}{N}{1} &= 1 \\
    \polynomialQ{1}{N}{2} &= 6N - 4 \\
    \polynomialQ{1}{N}{3} &= 18N - 27 \\
    \polynomialQ{1}{N}{4} &= 36N - 80 \\
    \polynomialQ{1}{N}{5} &= 60N - 175 \\
    \polynomialQ{1}{N}{6} &= 90N - 324 \\
    \polynomialQ{1}{N}{7} &= 126N - 539 \\
    \polynomialQ{1}{N}{8} &= 168N - 832 \\
    \polynomialQ{1}{N}{9} &= 216N - 1215 \\
    \polynomialQ{1}{N}{10} &= 270N - 1700 \\
    \polynomialQ{1}{N}{11} &= 330N - 2299 \\
    \polynomialQ{1}{N}{12} &= 396N - 3024 \\
    \polynomialQ{1}{N}{13} &= 468N - 3887 \\
    \polynomialQ{1}{N}{14} &= 546N - 4900 \\
    \polynomialQ{1}{N}{15} &= 630N - 6075 \\
    \polynomialQ{1}{N}{16} &= 720N - 7424 \\
    \polynomialQ{1}{N}{17} &= 816N - 8959 \\
    \polynomialQ{1}{N}{18} &= 918N - 10692 \\
    \polynomialQ{1}{N}{19} &= 1026N - 12635 \\
    \polynomialQ{1}{N}{20} &= 1140N - 14800
\end{align*}
\begin{figure}[H]
    \centering
    \includegraphics[width=1\textwidth]{sections/images/02_cubes_with_q_1_n_k}
    ~\caption{Polynomials Q(1, n, k)}\label{fig:figure2}
\end{figure}



    \subsection{Polynomials P(2,n,k)}
    \begin{align*}
    \polynomialP{2}{N}{0} &= 0 \\
    \polynomialP{2}{N}{1} &= 30N^2 - 60N + 31 \\
    \polynomialP{2}{N}{2} &= 150N^2 - 540N + 512 \\
    \polynomialP{2}{N}{3} &= 420N^2 - 2160N + 2943 \\
    \polynomialP{2}{N}{4} &= 900N^2 - 6000N + 10624 \\
    \polynomialP{2}{N}{5} &= 1650N^2 - 13500N + 29375 \\
    \polynomialP{2}{N}{6} &= 2730N^2 - 26460N + 68256 \\
    \polynomialP{2}{N}{7} &= 4200N^2 - 47040N + 140287 \\
    \polynomialP{2}{N}{8} &= 6120N^2 - 77760N + 263168 \\
    \polynomialP{2}{N}{9} &= 8550N^2 - 121500N + 459999 \\
    \polynomialP{2}{N}{10} &= 11550N^2 - 181500N + 760000 \\
    \polynomialP{2}{N}{11} &= 15180N^2 - 261360N + 1199231 \\
    \polynomialP{2}{N}{12} &= 19500N^2 - 365040N + 1821312 \\
    \polynomialP{2}{N}{13} &= 24570N^2 - 496860N + 2678143 \\
    \polynomialP{2}{N}{14} &= 30450N^2 - 661500N + 3830624 \\
    \polynomialP{2}{N}{15} &= 37200N^2 - 864000N + 5349375 \\
    \polynomialP{2}{N}{16} &= 44880N^2 - 1109760N + 7315456 \\
    \polynomialP{2}{N}{17} &= 53550N^2 - 1404540N + 9821087 \\
    \polynomialP{2}{N}{18} &= 63270N^2 - 1754460N + 12970368 \\
    \polynomialP{2}{N}{19} &= 74100N^2 - 2166000N + 16879999 \\
    \polynomialP{2}{N}{20} &= 86100N^2 - 2646000N + 21680000
\end{align*}
\begin{figure}[H]
    \centering
    \includegraphics[width=1\textwidth]{sections/images/03_fifth_power_with_p_1_n_k}
    ~\caption{Polynomials P(2, n, k)}\label{fig:figure3}
\end{figure}


    \subsection{Polynomials Q(2,n,k)}
    \begin{align*}
    \polynomialQ{2}{X}{0} &= 0 \\
    \polynomialQ{2}{X}{1} &= 1 \\
    \polynomialQ{2}{X}{2} &= 30X^2 - 60X + 32 \\
    \polynomialQ{2}{X}{3} &= 150X^2 - 540X + 513 \\
    \polynomialQ{2}{X}{4} &= 420X^2 - 2160X + 2944 \\
    \polynomialQ{2}{X}{5} &= 900X^2 - 6000X + 10625 \\
    \polynomialQ{2}{X}{6} &= 1650X^2 - 13500X + 29376 \\
    \polynomialQ{2}{X}{7} &= 2730X^2 - 26460X + 68257 \\
    \polynomialQ{2}{X}{8} &= 4200X^2 - 47040X + 140288 \\
    \polynomialQ{2}{X}{9} &= 6120X^2 - 77760X + 263169 \\
    \polynomialQ{2}{X}{10} &= 8550X^2 - 121500X + 460000 \\
    \polynomialQ{2}{X}{11} &= 11550X^2 - 181500X + 760001 \\
    \polynomialQ{2}{X}{12} &= 15180X^2 - 261360X + 1199232 \\
    \polynomialQ{2}{X}{13} &= 19500X^2 - 365040X + 1821313 \\
    \polynomialQ{2}{X}{14} &= 24570X^2 - 496860X + 2678144 \\
    \polynomialQ{2}{X}{15} &= 30450X^2 - 661500X + 3830625 \\
    \polynomialQ{2}{X}{16} &= 37200X^2 - 864000X + 5349376 \\
    \polynomialQ{2}{X}{17} &= 44880X^2 - 1109760X + 7315457 \\
    \polynomialQ{2}{X}{18} &= 53550X^2 - 1404540X + 9821088 \\
    \polynomialQ{2}{X}{19} &= 63270X^2 - 1754460X + 12970369 \\
    \polynomialQ{2}{X}{20} &= 74100X^2 - 2166000X + 16880000
\end{align*}
\begin{figure}[H]
    \centering
    \includegraphics[width=1\textwidth]{sections/images/04_fifth_power_with_q_1_n_k}
    ~\caption{Polynomials Q(2, n, k)}\label{fig:figure4}
\end{figure}


    \subsection{Polynomials P(3,n,k)}
    \begin{align*}
    \polynomialP{3}{X}{0} &= 0 \\
    \polynomialP{3}{X}{1} &= 140X^3 - 420X^2 + 406X - 125 \\
    \polynomialP{3}{X}{2} &= 1260X^3 - 7140X^2 + 13818X - 9028 \\
    \polynomialP{3}{X}{3} &= 5040X^3 - 41160X^2 + 115836X - 110961 \\
    \polynomialP{3}{X}{4} &= 14000X^3 - 148680X^2 + 545860X - 684176 \\
    \polynomialP{3}{X}{5} &= 31500X^3 - 411180X^2 + 1858290X - 2871325 \\
    \polynomialP{3}{X}{6} &= 61740X^3 - 955500X^2 + 5124126X - 9402660 \\
    \polynomialP{3}{X}{7} &= 109760X^3 - 1963920X^2 + 12182968X - 25872833 \\
    \polynomialP{3}{X}{8} &= 181440X^3 - 3684240X^2 + 25945416X - 62572096 \\
    \polynomialP{3}{X}{9} &= 283500X^3 - 6439860X^2 + 50745870X - 136972701 \\
    \polynomialP{3}{X}{10} &= 423500X^3 - 10639860X^2 + 92745730X - 276971300 \\
    \polynomialP{3}{X}{11} &= 609840X^3 - 16789080X^2 + 160386996X - 524988145 \\
    \polynomialP{3}{X}{12} &= 851760X^3 - 25498200X^2 + 264896268X - 943023888 \\
    \polynomialP{3}{X}{13} &= 1159340X^3 - 37493820X^2 + 420839146X - 1618774781 \\
    \polynomialP{3}{X}{14} &= 1543500X^3 - 53628540X^2 + 646725030X - 2672907076 \\
    \polynomialP{3}{X}{15} &= 2016000X^3 - 74891040X^2 + 965662320X - 4267591425 \\
    \polynomialP{3}{X}{16} &= 2589440X^3 - 102416160X^2 + 1406064016X - 6616398080 \\
    \polynomialP{3}{X}{17} &= 3277260X^3 - 137494980X^2 + 2002403718X - 9995653693 \\
    \polynomialP{3}{X}{18} &= 4093740X^3 - 181584900X^2 + 2796022026X - 14757360516 \\
    \polynomialP{3}{X}{19} &= 5054000X^3 - 236319720X^2 + 3835983340X - 21343778801 \\
    \polynomialP{3}{X}{20} &= 6174000X^3 - 303519720X^2 + 5179983060X - 30303773200
\end{align*}
\begin{figure}[H]
    \centering
    \includegraphics[width=1\textwidth]{sections/images/05_seventh_power_with_p_3_n_k}
    ~\caption{Polynomials P(3, n, k)}\label{fig:figure5}
\end{figure}


    \subsection{Polynomials Q(3,n,k)}
    \begin{align*}
    \polynomialQ{3}{N}{0} &= 0 \\
    \polynomialQ{3}{N}{1} &= 1 \\
    \polynomialQ{3}{N}{2} &= 140N^3 - 420N^2 + 406N - 124 \\
    \polynomialQ{3}{N}{3} &= 1260N^3 - 7140N^2 + 13818N - 9027 \\
    \polynomialQ{3}{N}{4} &= 5040N^3 - 41160N^2 + 115836N - 110960 \\
    \polynomialQ{3}{N}{5} &= 14000N^3 - 148680N^2 + 545860N - 684175 \\
    \polynomialQ{3}{N}{6} &= 31500N^3 - 411180N^2 + 1858290N - 2871324 \\
    \polynomialQ{3}{N}{7} &= 61740N^3 - 955500N^2 + 5124126N - 9402659 \\
    \polynomialQ{3}{N}{8} &= 109760N^3 - 1963920N^2 + 12182968N - 25872832 \\
    \polynomialQ{3}{N}{9} &= 181440N^3 - 3684240N^2 + 25945416N - 62572095 \\
    \polynomialQ{3}{N}{10} &= 283500N^3 - 6439860N^2 + 50745870N - 136972700 \\
    \polynomialQ{3}{N}{11} &= 423500N^3 - 10639860N^2 + 92745730N - 276971299 \\
    \polynomialQ{3}{N}{12} &= 609840N^3 - 16789080N^2 + 160386996N - 524988144 \\
    \polynomialQ{3}{N}{13} &= 851760N^3 - 25498200N^2 + 264896268N - 943023887 \\
    \polynomialQ{3}{N}{14} &= 1159340N^3 - 37493820N^2 + 420839146N - 1618774780 \\
    \polynomialQ{3}{N}{15} &= 1543500N^3 - 53628540N^2 + 646725030N - 2672907075 \\
    \polynomialQ{3}{N}{16} &= 2016000N^3 - 74891040N^2 + 965662320N - 4267591424 \\
    \polynomialQ{3}{N}{17} &= 2589440N^3 - 102416160N^2 + 1406064016N - 6616398079 \\
    \polynomialQ{3}{N}{18} &= 3277260N^3 - 137494980N^2 + 2002403718N - 9995653692 \\
    \polynomialQ{3}{N}{19} &= 4093740N^3 - 181584900N^2 + 2796022026N - 14757360515 \\
    \polynomialQ{3}{N}{20} &= 5054000N^3 - 236319720N^2 + 3835983340N - 21343778800 \\
\end{align*}
\begin{figure}[H]
    \centering
    \includegraphics[width=1\textwidth]{sections/images/06_seventh_power_with_q_3_n_k}
    ~\caption{Polynomials P(1, n, k)}\label{fig:figure6}
\end{figure}



%    \bibliographystyle{unsrt}
%    \bibliography{PlotsOfClosedForms}

\end{document}
